\PassOptionsToPackage{unicode=true}{hyperref} % options for packages loaded elsewhere
\PassOptionsToPackage{hyphens}{url}
%
\documentclass[10pt,ignorenonframetext,]{beamer}
\usepackage{pgfpages}
\setbeamertemplate{caption}[numbered]
\setbeamertemplate{caption label separator}{: }
\setbeamercolor{caption name}{fg=normal text.fg}
\beamertemplatenavigationsymbolsempty
% Prevent slide breaks in the middle of a paragraph:
\widowpenalties 1 10000
\raggedbottom
\setbeamertemplate{part page}{
\centering
\begin{beamercolorbox}[sep=16pt,center]{part title}
  \usebeamerfont{part title}\insertpart\par
\end{beamercolorbox}
}
\setbeamertemplate{section page}{
\centering
\begin{beamercolorbox}[sep=12pt,center]{part title}
  \usebeamerfont{section title}\insertsection\par
\end{beamercolorbox}
}
\setbeamertemplate{subsection page}{
\centering
\begin{beamercolorbox}[sep=8pt,center]{part title}
  \usebeamerfont{subsection title}\insertsubsection\par
\end{beamercolorbox}
}
\AtBeginPart{
  \frame{\partpage}
}
\AtBeginSection{
  \ifbibliography
  \else
    \frame{\sectionpage}
  \fi
}
\AtBeginSubsection{
  \frame{\subsectionpage}
}
\usepackage{lmodern}
\usepackage{amssymb,amsmath}
\usepackage{ifxetex,ifluatex}
\usepackage{fixltx2e} % provides \textsubscript
\ifnum 0\ifxetex 1\fi\ifluatex 1\fi=0 % if pdftex
  \usepackage[T1]{fontenc}
  \usepackage[utf8]{inputenc}
  \usepackage{textcomp} % provides euro and other symbols
\else % if luatex or xelatex
  \usepackage{unicode-math}
  \defaultfontfeatures{Ligatures=TeX,Scale=MatchLowercase}
\fi
\usetheme[]{UHBeamer}
% use upquote if available, for straight quotes in verbatim environments
\IfFileExists{upquote.sty}{\usepackage{upquote}}{}
% use microtype if available
\IfFileExists{microtype.sty}{%
\usepackage[]{microtype}
\UseMicrotypeSet[protrusion]{basicmath} % disable protrusion for tt fonts
}{}
\IfFileExists{parskip.sty}{%
\usepackage{parskip}
}{% else
\setlength{\parindent}{0pt}
\setlength{\parskip}{6pt plus 2pt minus 1pt}
}
\usepackage{hyperref}
\hypersetup{
            pdftitle={An Example R Markdown Document},
            pdfauthor={Steven V. Miller},
            pdfborder={0 0 0},
            breaklinks=true}
\urlstyle{same}  % don't use monospace font for urls
\newif\ifbibliography
\usepackage{color}
\usepackage{fancyvrb}
\newcommand{\VerbBar}{|}
\newcommand{\VERB}{\Verb[commandchars=\\\{\}]}
\DefineVerbatimEnvironment{Highlighting}{Verbatim}{commandchars=\\\{\}}
% Add ',fontsize=\small' for more characters per line
\usepackage{framed}
\definecolor{shadecolor}{RGB}{248,248,248}
\newenvironment{Shaded}{\begin{snugshade}}{\end{snugshade}}
\newcommand{\AlertTok}[1]{\textcolor[rgb]{0.94,0.16,0.16}{#1}}
\newcommand{\AnnotationTok}[1]{\textcolor[rgb]{0.56,0.35,0.01}{\textbf{\textit{#1}}}}
\newcommand{\AttributeTok}[1]{\textcolor[rgb]{0.77,0.63,0.00}{#1}}
\newcommand{\BaseNTok}[1]{\textcolor[rgb]{0.00,0.00,0.81}{#1}}
\newcommand{\BuiltInTok}[1]{#1}
\newcommand{\CharTok}[1]{\textcolor[rgb]{0.31,0.60,0.02}{#1}}
\newcommand{\CommentTok}[1]{\textcolor[rgb]{0.56,0.35,0.01}{\textit{#1}}}
\newcommand{\CommentVarTok}[1]{\textcolor[rgb]{0.56,0.35,0.01}{\textbf{\textit{#1}}}}
\newcommand{\ConstantTok}[1]{\textcolor[rgb]{0.00,0.00,0.00}{#1}}
\newcommand{\ControlFlowTok}[1]{\textcolor[rgb]{0.13,0.29,0.53}{\textbf{#1}}}
\newcommand{\DataTypeTok}[1]{\textcolor[rgb]{0.13,0.29,0.53}{#1}}
\newcommand{\DecValTok}[1]{\textcolor[rgb]{0.00,0.00,0.81}{#1}}
\newcommand{\DocumentationTok}[1]{\textcolor[rgb]{0.56,0.35,0.01}{\textbf{\textit{#1}}}}
\newcommand{\ErrorTok}[1]{\textcolor[rgb]{0.64,0.00,0.00}{\textbf{#1}}}
\newcommand{\ExtensionTok}[1]{#1}
\newcommand{\FloatTok}[1]{\textcolor[rgb]{0.00,0.00,0.81}{#1}}
\newcommand{\FunctionTok}[1]{\textcolor[rgb]{0.00,0.00,0.00}{#1}}
\newcommand{\ImportTok}[1]{#1}
\newcommand{\InformationTok}[1]{\textcolor[rgb]{0.56,0.35,0.01}{\textbf{\textit{#1}}}}
\newcommand{\KeywordTok}[1]{\textcolor[rgb]{0.13,0.29,0.53}{\textbf{#1}}}
\newcommand{\NormalTok}[1]{#1}
\newcommand{\OperatorTok}[1]{\textcolor[rgb]{0.81,0.36,0.00}{\textbf{#1}}}
\newcommand{\OtherTok}[1]{\textcolor[rgb]{0.56,0.35,0.01}{#1}}
\newcommand{\PreprocessorTok}[1]{\textcolor[rgb]{0.56,0.35,0.01}{\textit{#1}}}
\newcommand{\RegionMarkerTok}[1]{#1}
\newcommand{\SpecialCharTok}[1]{\textcolor[rgb]{0.00,0.00,0.00}{#1}}
\newcommand{\SpecialStringTok}[1]{\textcolor[rgb]{0.31,0.60,0.02}{#1}}
\newcommand{\StringTok}[1]{\textcolor[rgb]{0.31,0.60,0.02}{#1}}
\newcommand{\VariableTok}[1]{\textcolor[rgb]{0.00,0.00,0.00}{#1}}
\newcommand{\VerbatimStringTok}[1]{\textcolor[rgb]{0.31,0.60,0.02}{#1}}
\newcommand{\WarningTok}[1]{\textcolor[rgb]{0.56,0.35,0.01}{\textbf{\textit{#1}}}}
\usepackage{graphicx,grffile}
\makeatletter
\def\maxwidth{\ifdim\Gin@nat@width>\linewidth\linewidth\else\Gin@nat@width\fi}
\def\maxheight{\ifdim\Gin@nat@height>\textheight\textheight\else\Gin@nat@height\fi}
\makeatother
% Scale images if necessary, so that they will not overflow the page
% margins by default, and it is still possible to overwrite the defaults
% using explicit options in \includegraphics[width, height, ...]{}
\setkeys{Gin}{width=\maxwidth,height=\maxheight,keepaspectratio}
\setlength{\emergencystretch}{3em}  % prevent overfull lines
\providecommand{\tightlist}{%
  \setlength{\itemsep}{0pt}\setlength{\parskip}{0pt}}
\setcounter{secnumdepth}{0}

% set default figure placement to htbp
\makeatletter
\def\fps@figure{htbp}
\makeatother


\title{An Example R Markdown Document}
\providecommand{\subtitle}[1]{}
\subtitle{(A Subtitle Would Go Here if This Were a Class)}
\author{Steven V. Miller}
\date{13/12/2018}

\begin{document}
\frame{\titlepage}

\hypertarget{pop-songs-and-political-science}{%
\section{Pop Songs and Political
Science}\label{pop-songs-and-political-science}}

\hypertarget{morning-train}{%
\subsection{Morning Train}\label{morning-train}}

\begin{frame}{Sheena Easton and Game Theory}
\protect\hypertarget{sheena-easton-and-game-theory}{}

Sheena Easton describes the following scenario for her baby:

\begin{enumerate}
\tightlist
\item
  Takes the morning train
\item
  Works from nine 'til five
\item
  Takes another train home again
\item
  Finds Sheena Easton waiting for him
\end{enumerate}

\bigskip Sheena Easton and her baby are playing a
\textcolor{orange}{zero-sum (total conflict) game}.

\begin{itemize}
\tightlist
\item
  Akin to Holmes-Moriarty game (see: von Neumann and Morgenstern)
\item
  Solution: \textcolor{orange}{mixed strategy}
\end{itemize}

\end{frame}

\hypertarget{never-gonna-give-you-up}{%
\subsection{Never Gonna Give You Up}\label{never-gonna-give-you-up}}

\begin{frame}{Rick Astley's Re-election Platform}
\protect\hypertarget{rick-astleys-re-election-platform}{}

Rick Astley's campaign promises:

\begin{itemize}
\tightlist
\item
  Never gonna give you up
\item
  Never gonna let you down
\item
  Never gonna run around and desert you
\item
  Never gonna make you cry
\item
  Never gonna say goodbye
\item
  Never gonna tell a lie and hurt you.
\end{itemize}

\bigskip Whereas these pledges conform to the preferences of the
\textcolor{orange}{median voter}, we expect Congressman Astley to secure
re-election.

\end{frame}

\hypertarget{caribbean-queen}{%
\subsection{Caribbean Queen}\label{caribbean-queen}}

\begin{frame}{Caribbean Queen and Operation Urgent Fury}
\protect\hypertarget{caribbean-queen-and-operation-urgent-fury}{}

In 1984, Billy Ocean released ``Caribbean Queen''.

\begin{itemize}
\tightlist
\item
  Emphasized sharing the same dream
\item
  Hearts beating as one
\end{itemize}

\bigskip ``Caribbean Queen'' is about the poor execution of Operation
Urgent Fury.

\begin{itemize}
\tightlist
\item
  Echoed JCS chairman David Jones' frustrations with military
  establishment.
\end{itemize}

\bigskip Billy Ocean is advancing calls for what became the
Goldwater-Nichols Act.

\begin{itemize}
\tightlist
\item
  Wanted to take advantage of \textcolor{orange}{economies of scale},
  resolve \textcolor{orange}{coordination problems} in U.S. military.
\end{itemize}

\end{frame}

\hypertarget{good-day}{%
\subsection{Good Day}\label{good-day}}

\begin{frame}{The Good Day Hypothesis}
\protect\hypertarget{the-good-day-hypothesis}{}

We know the following about Ice Cube's day.

\begin{enumerate}
\tightlist
\item
  The Lakers beat the Supersonics.
\item
  No helicopter looked for a murder.
\item
  Consumed Fatburger at 2 a.m.
\item
  Goodyear blimp: ``Ice Cube's a pimp.''
\end{enumerate}

\bigskip This leads to two different hypotheses:

\begin{itemize}
\tightlist
\item
  \(H_0\): Ice Cube's day is statistically indistinguishable from a
  typical day.
\item
  \(H_1\): Ice Cube is having a good (i.e.~greater than average) day.
\end{itemize}

\bigskip These hypotheses are tested using archival data of Ice Cube's
life.

\end{frame}

\hypertarget{rendering-this-document}{%
\section{Rendering This Document}\label{rendering-this-document}}

\begin{frame}{The Problem of Rendering in Markdown}
\protect\hypertarget{the-problem-of-rendering-in-markdown}{}

One big disadvantage to Markdown: compiling.

\bigskip Here's what it would look like from Terminal \medskip

\begin{figure}
\centering
\includegraphics{uhasselt_logo.png}
\caption{Markdown Call}
\end{figure}

\bigskip Nobody got time for that.

\end{frame}

\begin{frame}{One Alternative: RStudio}
\protect\hypertarget{one-alternative-rstudio}{}

\begin{center}
  \includegraphics[width=1.00\textwidth]{uhasselt_logo.png}
\end{center}

\end{frame}

\begin{frame}[fragile]{Another Alternative: Rscript}
\protect\hypertarget{another-alternative-rscript}{}

Another option: noninteractive \texttt{Rscript}

\begin{itemize}
\tightlist
\item
  I prefer this option since I tend to not like GUIs.
\item
  Assumes you're on a Linux/Mac system.
\end{itemize}

Save this to a .R script (call it whatever you like)

\begin{itemize}
\tightlist
\item
  Note that the ``s'' in ``utils'' package is cut off in verbatim
  environment below.
\end{itemize}

\begin{Shaded}
\begin{Highlighting}[]
\CommentTok{#! /usr/bin/Rscript --vanilla --default-packages=base,stats,utils}
\KeywordTok{library}\NormalTok{(knitr)}
\KeywordTok{library}\NormalTok{(rmarkdown)}
\NormalTok{file <-}\StringTok{ }\KeywordTok{list.files}\NormalTok{(}\DataTypeTok{pattern=}\StringTok{'.Rmd'}\NormalTok{)}
\NormalTok{rmarkdown}\OperatorTok{::}\KeywordTok{render}\NormalTok{(file)}
\end{Highlighting}
\end{Shaded}

Make it executable. Double click or run in Terminal.

\begin{itemize}
\tightlist
\item
  Keep a copy in each directory, but keep only one .Rmd per directory.
\end{itemize}

\end{frame}

\hypertarget{conclusion}{%
\section{Conclusion}\label{conclusion}}

\begin{frame}{Conclusion}
\protect\hypertarget{conclusion-1}{}

Beamer markup is messy. Markdown is much more elegant.

\begin{itemize}
\tightlist
\item
  Incorporating R with Markdown makes Markdown that much better.
\item
  Rendering Markdown \(\rightarrow\) Beamer requires minimal Rscript
  example.
\item
  I provide such a script to accompany this presentation.
\end{itemize}

\end{frame}

\end{document}
